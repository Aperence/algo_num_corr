\documentclass[A4paper,11pt]{article}
\usepackage[utf8]{inputenc}
\usepackage{graphicx}
\usepackage{geometry}
\usepackage[dvipsnames]{xcolor}
\usepackage{hyperref}
\usepackage{diagbox}
\usepackage{amsmath}
\usepackage{wrapfig}
\usepackage{listings} %code highlighter
\usepackage{color} %use color
\usepackage{subcaption}
\usepackage{float}


\geometry{
 a4paper,
 total={150mm,240mm},
 left=30mm,
 top=20mm,
 }
\title{Algorithmique numérique : Examen Janvier 2021-2022}
\author{Retranscrit par Jacques Hogge, correction par Doeraene Anthony}

\begin{document}
\maketitle


\begin{enumerate}
    \item Question 1
    
Etant donné le nombre 28.375

a) Quelle est la représentation binaire à points fixe du nombre avec 8bits(k=8) pour la partie frationnaire. le format attendu est XXXXXXXX.XXXXXXXX avec X=0 ou 1.

\textbf{ 0001 1100.0110 0000}

b) Quelle serait la valeur du bit31 (1bit) en utilisant une représentation float32

\textbf{0 car positif}

c) Quelle serait la valeur des bit22 à 14 (8bits) en utilisant une représentation float32

\textbf{1100 0110 car bits de mantisse prend représentation en float de question a = 11100.011 et on shift jusqu'à n'avoir plus qu'un bit : 1.1100011 => la mantisse est donc 11000110 (en comblant le dernier bit)}

d) Quelle serait la valeur des bit30 à 23 (8bits) en utilisant une représentation float32

\textbf{1000 0011 : à partir de c, on a du shifter de 4 à droite : doit faire donc $*2^4$. La valeur de l'exposant est donc $x-127 = 4 <=> x = 131$ et $131 = 128 + 2 + 1$}

e) Dans ce cas particulier, quelle est l'erreur absolue sur cette représentation float32

\textbf{ 0 : on représente exactement le nombre}


\item Question 2

Etant donné la fonction suivante avec x,y et z des nombres positif et avec x et y ayant généralement des valeurs proches:

def f(x,y,z): return np.log(np.exp(x) np.exp(-y) np.exp(z))

a) Améliore la stabilité/ qualité numérique du code

\textbf{def f(x,y,z) : return x+z-y    ; (doit éviter la soustraction entre deux nombres proches)}

b) Quelle est l'erreur relative maximale liée au arrondi sur l'opérateur x⊗y en float32

 \begin{flalign*}
  & e_f = y * e_x + x * e_y  \\
  & e_f = x*y*\epsilon_x + x*y*\epsilon_y \\
  & e_f = x*y*(\epsilon_x + \epsilon_y)\\
  & \epsilon_f = (\epsilon_x + \epsilon_y)
 \end{flalign*}
 L'erreur maximale est donc, en notant $\epsilon_x = \epsilon_y = 2^{-p}$ et p = 24 car travaillant en float32 : $2^{-24} + 2^{-24} = 2^{-23}$ en rajoutant l'erreur de float on arrive au même résultat
 
\item Question 3

La vitesse d'une fusée est mesurée a trois moment différents :(t1,t2,t3
). Nous voulons interpoler par un polynôme la vitesse de la fusée au temps t en utilisant ces trois mesures (v1,v2,v3).
Ecris le polynôme de Lagrange P(t)
. La solution doit avoir un forme algébrique (pas numérique) et ne faire intervenir que les variables : t,t1,t2,t3,v1,v2,v3
\[
P(t) = \frac{(t - t_2)(t - t_3)}{(t_1 - t_2)(t_1 - t_3)}*v_1 + \frac{(t - t_1)(t - t_3)}{(t_2 - t_1)(t_2 - t_3)}*v_2 + \frac{(t - t_1)(t - t_2)}{(t_3 - t_1)(t_3 - t_2)}*v_3
\]


\item Question 4

Si les mesures de vitesse de la fusée de la question 3 sont celles-ci-dessous, quel étais la vitesse de la fusée a t=2.5

\begin{center}
    \begin{tabular}{c|ccc}
     t & 1.0 & 2.0 & 3.0\\ \hline
     v(t) & 1.0 & 2.0 & 5.0
\end{tabular}

\end{center}

Marche a suivre : Utilise une interpolation cubique-spline "naturelle"

a) quelle est la matrice (tri-diagonal) associée au système des courbures


\begin{pmatrix}
    1 & 0 & 0\\
    1 & 4 & 1 \\
    0 & 0 & 1
\end{pmatrix}

b) quelle sont les courbures de la cubic-spline

\textbf{cubiques-spline naturelle donc $k_1 = k_3 = 0$. En appliquant ensuite la formule, 
$k_1 + 4*k_2 + k_3 = 12$ et donc $k_2 = 3$}

c) quelle est la vitesse de la fusée au moment t=2.5

\textbf{Via la formule en remplaçant les valeurs : 3.3125}

\item Question 5

Completes les bouts de code manquant pour rendre opérationnel cet algo qui localise les racines de la fonction f
par la méthode de bissection avec un précisions supérieur ou égal au parametre 'tol'. La racine de f
se trouve a priori entre x1 et x2
\begin{lstlisting}
import math
from numpy import sign
def bisection(f,x1,x2,tol=1.0e-9):
    f1 = f(x1)
    f2 = f(x2)
    if f1 == 0.0 : return x1
    if f2 == 0.0 : return x2
    if sign(f1) == sign(f2):
        raise Exception('Root is not brackted')
    n = # a) completer le code manquant
    
    for i in range(n):
        x3 = # b) completer le code manquant
        f3 = f(x3)
        if f3 = 0.0 : return x3
        if sign(f2) != sign(f3):
            # c) completer le code manquant
        else:
            # d) completer le code manquant
        return (x1,x2)/2.0
        
    a) int(math.ceil(np.log(abs(x2 - x1)/tol) / np.log(2)))
    b) (x1+x2)/2 ; f3 = f(x3)
    c) x1 = x3; f1 = f3
    d) x2 = x3; f2 = f3
\end{lstlisting}


\item Question 6

Dérive l'approximation par différence finie vers l'avant de deuxieme ordre (second order forward finite difference approximation) de f′(x)
à partir des expansions de taylors de f auteur de x

a) Quelles sont les deux expansions de Taylor nécessaire à ce développement
\[
f(x+h) = f(x) + h*f'(x) + h^2 * \frac{f^{(2)}(x)}{2!} + O(h^3)
\]
\[
f(x+h2) = f(x) + 2h*f'(x) + 4* h^2 * \frac{f^{(2)}(x)}{2!} + O(h^3)
\]
b) quelle est l'approximation de f′(x) dans ce cas

\textbf{En éliminant $f^{(2)}$ en faisant $f(x+2h) - 4f(x+h)$ et en isolant $f'(x)$}
\[
f'(x) = \frac{-3*f(x) + 4*f(x+h) - f(x+2h)}{2h}
\]
c) quel est l'ordre de l'erreur sur cette aprroximation (en terme de h)

\textbf{$O(h^2)$ Car forward second en voyant l'ordre restant après avoir isolé $f'(x)$}

\item Question 7

Etant données les données suivantes:
\begin{center}
    \begin{tabular}{c|ccccc}
         x & 3.0 & 4.0 & 5.0 & 6.0 & 7.0  \\ \hline
         f(x) & 1.2 & 2.3 & 3.5 & 4.7 & 5.9 
    \end{tabular}
\end{center}

a) Estime f′(x) en x=3.0 en utilisant la formule dérivée à la question précédente

\textbf{En remplaçant dans la formule avec $h=1$ on obtient $f'(3) = 1.05$ (ou en utilisant $h=2$, on obtient $f'(3) = 1.125$)}

b) Estime f′(x) en x=3.0 en utilisant la méthode la plus précise possible

\textbf{En utilisant Richardson avec les deux approxiamtions précédantes, on trouve $f'(3) = \frac{4* 1.05 - 1.125}{3} = 1.025$}

c) Dans le deuxieme cas, quel est l'ordre de l'erreur sur l'approximation (en terme de h)

\textbf{O(h^4)}

\item Question 8

Etant données les données de la question 7:

a) Quelle est l'intégrale de la fonction f(x) entre 3 et 7 en utilisant Newton-Cotes avec n=1 (et la méthode composite si possible)

\begin{flalign*}
&I = \frac{h}{2}* (f(3) + 2f(4) + 2f(5) + 2f(6) + f(7))\\
&I = \frac{1}{2} * (28.1)\\
&I = 14.05
\end{flalign*}

b) Quel est l'ordre de l'erreur commise sur cette intégral en terme de h

\textbf{O(h^2)}

c) quelle méthode pourrait-on utiliser pour réduire l'erreur et quel serait alors l'ordre de l'erreur réduite

\textbf{Romberg avec 5 points = 4 panels. On peut déterminer k en posant via les formules $2^{k-1}=panels$, on trouve donc $k = 3$ et l'ordre de l'erreur est donc $O(h^{2*k}) = O(h^6)$}

\item Question 9

Nous souhaitons résoudre le problème de Cauchy (Initial Value Problem) suivant y′′=−y−0.1y′
avec les conditions initials y(0)=0 et y′(0)=1 .Nous pouvons utiliser la méthode RK4 qui a comme paramètre F, xstart, xstop, ystart et h. Ecris la fonction F(y,x) que nous devons utiliser pour résoudre ce problème

\vspace{4px}
\begin{lstlisting}
def F(y,x):
    return np.array([ y[1], 
                      -y[0] - 0.1*y[1]])
\end{lstlisting}

\item Question 10

Nous voulons optimiser les variables de design x,y afin de minimiser le coût associé.

Le cout est proportionnel à $y+(y−2)^2+5yx$

En plus nous devons respecter les contraites suivantes : 
\begin{enumerate}
    \item $x \geq 0$
    
    \item $1 \leq y \leq 10$
    
    \item $x+y=10$
\end{enumerate}


a) Ecris la fonction objective que nous devons passer a l'algo Powell pour résoudre ce problème. la variable Lambda = 1 (definie globalement) pour etre utiliseé pour modérer l'impact d'une éventuelle pénalité

\begin{lstlisting}
def f(x,y):
    return y + (y-2.0)**2 + 5*x*y + Lambda*(min(x,0)**2 
    + max(y-10, 0)**2 + min(y-1, 0)**2 + (x+y-10)**2)
\end{lstlisting}

b) Quelle procédure adopter si les variables de design optimal violent les contraintes ?

\textbf{Augmenter la valeur de Lambda et recommencer l'optimisation à partir du précédent point optimal trouvé}
\end{enumerate}



\end{document}
